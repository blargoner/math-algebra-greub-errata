% Errata from Linear Algebra and Multilinear Algebra by Greub
% By John Peloquin
\documentclass[letterpaper,12pt]{article}
\usepackage{amsmath,amssymb,amsthm,enumitem,fourier,stmaryrd}

\renewcommand{\H}{\mathbb{H}}

\newcommand{\from}{\leftarrow}

\DeclareMathOperator{\tr}{tr}

\newcommand{\union}{\cup}
\newcommand{\after}{\circ}
\newcommand{\bigdsum}{\bigoplus}
\newcommand{\tprod}{\otimes}
\newcommand{\bigtprod}{\bigotimes}

\newcommand{\sprod}[2]{\langle#1,#2\rangle}
\newcommand{\iprod}[2]{(#1,#2)}
\newcommand{\oc}[1]{#1^{\perp}}

% Meta
\title{Errata from \\\textit{Linear Algebra} and \textit{Multilinear Algebra}}
\author{John Peloquin}
\date{}

\begin{document}
\maketitle

% Intro
\section*{Introduction}
This document contains errata from \cite{greub1} and~\cite{greub2}. Locations within each text are indicated by coordinates~\((p,n)\), where \(p\)~is a page number and \(n\)~is a line number on page~\(p\). Positive line numbers count from the top of the page, whereas negative line numbers count from the bottom of the page. Displayed equations, diagrams, and figures are counted as single lines.

% Linear algebra
\newpage
\part*{Linear Algebra}
Errata are currently only listed for Chapters 0--VIII (except \S~3 of Chapter~VI and \S~7 of Chapter~VIII) and Chapter~XI (except \S~4--5).

Minor, purely typographical inconsistencies like those between ``\(\to\)'' and ``\(\mapsto\)'', or between ``\(1\ldots n\)'' and ``\(1,\ldots,n\)'', or between ``family \(x_{\alpha}\)'' and ``family \((x_{\alpha})\)'' and ``family \(\{x_{\alpha}\}\)'', are not listed but occur frequently.

\section*{Chapter~0}
\begin{itemize}
\item (2, 8): in the definition of subgroup, ``subset~\(H\)'' should be ``nonempty subset~\(H\)''.
\item (3, -12): a homomorphism between fields must also preserve the multiplicative identity.
\item (3, -13): a subfield must also contain the multiplicative identity.
\end{itemize}

\section*{Chapter~I}
\begin{itemize}
\item (9, 16): in the definition of linear dependence, ``non-trivial linear combination of the~\(x_{\alpha}\)'' should be ``non-trivial linear combination of the~\(x_{\alpha}\) equal to zero''.
\item (12, 1): \(\lambda^i=0\) should be \(\lambda^i\ne 0\).
\item (12, 5): throughout the proof of~(ii), \(n\)~should be~\(m\).
\item (12, -5): ``The a partial order'' should be ``A partial order''.
\item (12, -4): ``maximal element'' should be ``upper bound''.
\item (13, 8): \(x\in E\) should be \(x\in S-T\).
\item (13, 9), (13, 10): \(T\union x\) should be \(T\union\{x\}\) and \(x\union T\) should be \(\{x\}\union T\).
\item (14, 1): ``element~\(f_a\)'' should be ``elements~\(f_a\)''.
\item (14, 3), (14, 5): in the displayed equations, \(j=1\) should be \(i=1\).
\item (15, -2): in problem~10, \(\{x_{\alpha}\}_{\alpha\ne\beta}\) should be \(\{a\}\union\{x_{\alpha}\}_{\alpha\ne\beta}\).
\item (20, 16): ``\(\varphi(S)\)~is a system of generators for~\(\varphi(S)\)'' should be ``\(\varphi(S)\)~is a system of generators for~\(F\)''.
\item (22, -15): in part~(v) of Problem~5, the concept of a generated subspace has not yet been defined.
\item (23, 7): in the definition of subspace, ``subset of~\(E\)'' should be ``nonempty subset of~\(E\)''.
\item (27, -11): in the displayed equation, \(\lambda_i\)~should be~\(\lambda^i\) (two occurrences). Also, it should be noted that \(y_i\in E_1\) and \(z_i\in E_2\).
\item (29, -16): ``\emph{canonical projection} of~\(E\) onto~\(E_1\)'' should be ``\emph{canonical projection} of~\(E\) onto~\(E/E_1\)''.
\item (40, -8): in the second displayed equation in problem~6, \(u\in E_2'\) should be \(y\in E_2'\).
\end{itemize}

\section*{Chapter~II}
\begin{itemize}
\item (47, -6): ``assume that there'' should be ``assume that''.
\item (52, -8): \(\psi:E_1\from F\) should be \(\psi:E\from F\).
\item (52, -3): ``left inverse'' should be ``left inverse~\(\psi\)''.
\item (53, 13): ``inverse'' should be ``left inverse''.
\item (53, -1): in problem~1, the inclusion \(L(E;F)\subset C(E;F)\) is wrong.
\item (59, -5): in~(2.25), it should be noted that \(\delta_{\varrho\sigma}\)~is a Kronecker delta.\footnote{This notation is defined later in a footnote on p.~76.}
\item (62, 10): in~(2.34), \(y\)~should be~\(y_j\).
\item (63, -7): in problem~7, ``second set'' should be ``disjoint set''.
\item (67, -1): in the displayed equation, \(y^*\in F\) should be \(y^*\in F^*\).
\item (68, -7): in the displayed equation, \(\langle y^*(\varphi+\psi)x\rangle\) should be \(\sprod{y^*}{(\varphi+\psi)x}\).
\item (76, 9): ``imension'' should be ``dimension''.
\item (76, -7): in the displayed equation (and really the rest of subsection~2.31), \(\varphi^{\mu}_{\nu}\)~should be~\(\varphi^{\nu}_{\mu}\).
\item (77, -7): \(\varphi\)~should be~\(\Phi\).
\end{itemize}

\section*{Chapter~III}
\begin{itemize}
\item (83, -10): in the displayed equation, \((b^{\mu}_1\,\ldots\,b^{\mu}_n)\) should be \((\alpha^{\mu}_1\,\ldots\,\alpha^{\mu}_n)\).
\item (84, -7): ``columns of the matrix~\(\alpha^{\mu}_{\nu}\)'' should be ``columns of the matrix~\(\alpha^{*\mu}_{\nu}\)''.
\item (84, -6): \(y=y^{*\mu}\) should be \(y^*=y^{*\mu}\).
\item (85, 13): in the displayed equation, it should be noted that \(A=(\alpha^{\mu}_{\nu})\).
\item (88, 1): in the main theorem, ``system of \(n\) equations in \(m\) unknowns'' should be ``system of \(m\) equations in \(n\) unknowns''.
\item (89, 12): it should be noted that \(\dim E=n\) and \(\dim F=m\).
\item (89, -8): in the displayed equation, \(a^{\mu}_{\nu}\)~should be~\(\alpha^{\mu}_{\nu}\).
\item (91, 5): ``\(E\) automorphism of~\(E\)'' should be ``automorphism of~\(E\)''.
\item (91, 9): in the displayed equation, \(M(\varphi)^{-1}\)~should be~\(M(\varphi^{-1})\).
\item (91, -3): in problem~3, ``linear transformation'' should be ``a linear transformation''.
\item (93, -13): ``inverse of the matrix of the transformation \(x_{\nu}\to\bar{x}_{\nu}\)'' should be ``transpose of the inverse of the matrix of the transformation \(x_{\nu}\to\bar{x}_{\nu}\)''.
\item (95, 12): \(\mu=1,\ldots,n\) should be \(\mu=1,\ldots,m\).
\item (96, 1): multiplication of basis vectors by nonzero scalars should be added to the list of elementary basis transformations.
\item (96, -7): \(2\leqq\nu\leqq m\) should be \(2\leqq\nu\leqq n\).
\item (98, 3): (3.36)~should be
\[\xi^r=(\kappa^r_r)^{-1}\left(\omega^r-\sum_{\nu=r+1}^n\kappa^r_{\nu}\xi^{\nu}\right)\]
\end{itemize}

\section*{Chapter~IV}
\begin{itemize}
\item (103, 15): in~(4.6), it should be noted that \(\hat{x}_j\)~indicates that the vector~\(x_j\) is deleted.\footnote{This notation is defined later in a footnote on p.~198.}
\item (105, -8): ``Proposition~II'' should be ``Proposition~III''.
\item (106, 5): ``(4.14)'' should be ``(4.12)''.
\item (107, 13): in~(4.14), \(p=\dim E\) should be \(p=\dim E_1\).
\item (109, 1): the problem numbers on this page should be incremented by~\(1\).
\item (109, 10): in problem~6, it must be assumed that \(E\)~is real.
\item (109, -11): in problem~8, the trace of a linear transformation has not yet been defined.
\item (113, 1): ``(4.21)'' should be ``(4.22)''.
\item (113, -10): in the displayed equation, \({x^*}_i\)~should be~\(x^{*i}\).
\item (115, 1): ``(4.14)'' should be ``(4.12)''.
\item (115, 4): in the displayed equation, \(\widehat{\varphi a_j}\)~should be~\(\widehat{\varphi a_i}\).
\item (115, 9): it should be noted that \(M(\varphi)=(\alpha^{\mu}_{\nu})\).
\item (115, -10): the displayed equation should be \(\beta^i_j=\det C^j_i\).
\item (116, 11): ``(4.36)'' should be ``(4.38)''.
\item (116, -1): in the displayed matrix~\(B^j_i\), the first column should have entries \(1,\alpha^j_1,\ldots,\alpha^j_{i-1},\alpha^j_{i+1},\ldots,\alpha^j_n\).
\item (117, 3): ``(4.38)'' should be ``(4.16)'', or a reference to Problem~6.
\item (117, 7): ``(4.35) and (4.30)'' should be ``(4.38) and (4.40)''.
\item (117, -14): ``minor'' should be ``submatrix''.
\item (122, 2): on the third line of the displayed equation, \(\sigma\)~should be~\(\varrho\).
\item (130, -14): in problem~11, it must be assumed that \(E\)~is real.
\item (135, -3): ``4.30'' should be ``4.29''.
\item (136, -6): ``4.17'' should be ``4.16'', and it should be noted that \(\xi^i_{\nu}\)~are the components of~\(x_{\nu}\) with respect to some basis.
\item (139, -13): \((a_1',b_2\ldots b_n)\) should be \((a_1',b_3\ldots b_n)\).
\item (139, -10): in the displayed equation, \(i=1\ldots n\) should be \(i=1\ldots n-1\).
\end{itemize}

\section*{Chapter~V}
\begin{itemize}
\item (146, 1): \((\varphi,\psi)\to\psi\varphi\) should be \((\psi,\varphi)\to\psi\varphi\).
\item (148, 10): the displayed list should also include elements of the form \(asb\) for \(s\in S\) and \(a,b\in A\).
\item (148, -17): it must be assumed that \(A\ne 0\) (equivalently \(e\ne 0\)) for it to follow that \(\lambda=0\).
\item (151, 2): the extra ``can be'' should be deleted.
\item (158, 14): ``cheeked'' should be ``checked''.
\item (160, -13): ``let be'' should be ``be''.
\item (160, -8): it must be assumed that \(E\ne 0\) for \(A(E;E)^2\ne 0\).
\item (161, -1): ``non-zero, \(I\), ideal'' should be ``non-zero ideal~\(I\),''.
\item (165, 12), (166, 7): it should be clarified that \(\Gamma\) is assumed to be a field under the restrictions of the operations in \(A_{\Delta}(E;E)\).
\end{itemize}

\section*{Chapter~VI}
\begin{itemize}
\item (168, -16): it should be noted that the zero map is homogeneous of every degree (hence \emph{the} degree is not well-defined in that case).
\item (172, -4): ``product'' should be ``products''.
\item (173, -3): in problem~6, \(\deg\varphi^*=-k\).
\item (175, 4): the displayed statement should be \(xe_k\in A_{l+k}\).
\item (176, 1): \(E\)~should be~\(A\).
\item (177, 2): in the second part of problem~1, it must be assumed that \(A\ne 0\) to conclude that \(k=0\).
\item (177, 14): in problem~5, ``\(\leqq 0\) (\(\geqq 0\))'' should be ``\(\geqq 0\) (\(\leqq 0\))''.
\end{itemize}

\section*{Chapter~VII}
\begin{itemize}
\item (189, -12): in the displayed equation, \(\lambda>0\) should be \(\lambda\geqq 0\).
\item (192, 5): \(x=x_{\mu}\) should be \(y=x_{\mu}\).
\item (192, 14): ``basisvectors'' should be ``basis vectors''.
\item (193, 14): ``orthogonal bases'' should be ``orthonormal bases''.
\item (193, -8): \((\alpha^{\mu}_{\varphi})\)~should be~\((\alpha^{\mu}_{\nu})\).
\item (195, 4): in problem~3, it should be assumed that \(E\)~is finite-dimensional or else established that orthogonal projection still works as long as \(E_1\)~is finite-dimensional.
\item (204, 13): in the displayed equation, \(\sum_{\nu}g_{\nu\lambda}\xi^{\lambda}\) should be \(\sum_{\nu}g_{\nu\lambda}\xi^{\nu}\).
\item (206, 14): ``least upper bound'' should be ``least nonnegative upper bound'' to account for the case \(E=0\) where there are no unit vectors.
\item (207, 7): ``naturaltopology'' should be ``natural topology''.
\item (210, 15): throughout subsection~7.24, it should be assumed that \(A\ne 0\).
\end{itemize}

\section*{Chapter~VIII}
\begin{itemize}
\item (217, 13): the concept of a metric tensor has not yet been defined.
\item (217, 14): in the displayed equation, \(\displaystyle\sum_{\nu}\)~should be~\(\displaystyle\sum_{\lambda}\).
\item (220, -1): in problem~1, equality holds only if \(\psi=\lambda\varphi\) or \(\varphi=\lambda\psi\).
\item (223, -2): in~(8.22), \(\lambda-\lambda_i\) should be \(\lambda_i-\lambda\).
\item (226, 17): in problem~11, the concept of a rotation has not yet been defined.
\item (227, 14): \(E_1\in\oc{E_2}\) should be \(E_1\subset\oc{E_2}\).
\item (231, 7): the proof of the normal form~(8.35) is incorrect because it is not true in general that the~\(a_n\) defined form an orthonormal basis of the space.
\item (231, -5): ``Proposition~II'' should be ``Proposition~III''.
\item (234, 4): in the displayed equation, \(a^{\kappa}_{\mu}\)~should be~\(\alpha^{\kappa}_{\mu}\).
\item (234, -16): ``\emph{rotation}'' should be ``\emph{rotation}~\(\varphi\)''.
\item (234, -10): it must be assumed that \(e\ne 0\).
\item (234, -9): ``sec.~4, 17'' should be ``sec.~4.17''.
\item (238, -7): in~(8.40), the second equation should be \(\sin\theta=-\tfrac{1}{2}\tr(j\after\varphi)\).
\item (239, 17): in the displayed equation, ``\(\equiv\)''~should be~``\(=\)''.
\item (240, -13): ``see~8.21'' should be ``sec.~8.21''.
\item (241, 6): in~(8.49), \(x\)~should be~\(u\).
\item (242, -13): in the displayed equation, ``\(\equiv\)''~should be~``\(=\)''.
\item (243, 7): \(E\)~should be~\(e\).
\item (243, -5): ``see~8.21'' should be ``sec.~8.23''.
\item (243, -4): in the displayed equation, \((z,\tau,z)\) should be~\(\iprod{z}{\tau z}\).
\item (243, -2): it should be noted that \(\Delta\)~is the positive normed determinant function in~\(E_1\).
\item (244, -11): \(F_1\)~should be~\(F\).
\item (245, 8): in the displayed equation, \(b\)~should be~\(b_1\).
\item (245, -10), (245, -5): ``(8.53)'' should be ``(8.54)''.
\item (246, 10), (246, -10): in problems 3 and~5, ``plane'' should be ``plane~\(E\)''.
\item (247, -1): \(f\)~should be defined so that \(f(0)=0\).
\item (248, -8): in problem~16, \(E\)~should be~\(\H\).
\end{itemize}

\section*{Chapter~XI}
\begin{itemize}
\item (328, 14), (328, -7): the proof of the equality condition for the triangle inequality does not properly account for the case \(x=0\); in particular, ``\(y=\lambda x\)'' should be ``\(y=\lambda x\) or \(x=\lambda y\)''.
\item (332, 11): in problem~3, the displayed equation should be
\[\iprod{z_1}{z_2}=\bigl(\iprod{x_1}{x_2}+\iprod{y_1}{y_2}\bigr)+i\bigl(\iprod{x_2}{y_1}-\iprod{x_1}{y_2}\bigr)\]
and it should be noted that \(z_k=x_k+iy_k\).
\item (334, -7): ``automorphism'' should be ``involution''.
\item (335, -12): in the displayed equation, \(\Phi\)~should be~\(\varphi\).
\item (335, -5): the concept of a conjugate determinant function is not defined.
\item (335, -3): in the displayed equation, \(\bar{x}_p\)~should be~\(\bar{x}_n\).
\item (336, -8): in the displayed equation, \(\displaystyle\sum_{\mu,\nu}\)~should be~\(\displaystyle\sum_{\mu,\lambda}\).
\item (337, -16): ``normal mapping'' should be ``normal mappings''.
\item (337, -10): in the displayed equation, \(\lambda\)~should be~\(\lambda_1\).
\item (339, 8): it must be assumed that \(e\ne 0\).
\item (339, 14): in problem~1, ``bilinear'' should be ``binary''.
\end{itemize}

% Multilinear algebra
\newpage
\part*{Multilinear Algebra}
Errata are currently only listed for Chapters~1--3 (except \S~4,8--17 of Chapter~2). As above, minor purely typographical inconsistencies are not listed.

\section*{Chapter~1}
\begin{itemize}
\item (4, 11): in problem~3, it should be assumed that \(E\ne 0\).
\item (4, -7): in problem~5(b), the claim is false.
\item (7, -5): in the displayed equation, \(\lambda^{\nu}_{\alpha}\in\Gamma\) should be \(\lambda^{\alpha}_{\nu}\in\Gamma\).
\item (11, -10): ``satisfis'' should be ``satisfies''.
\item (11, -3): The function name~\(f\) should not be reused.
\item (12, 9): in problem~1,
\[(\xi^1,\ldots,\xi^n)\times(\eta^1,\ldots,\eta^m)\]
should be
\[\bigl(\,(\xi^1,\ldots,\xi^n),(\eta^1,\ldots,\eta^m)\,\bigr)\]
\item (12, -6): in problem~6(a), the tensor product of linear maps has not yet been defined.
\item (13, 15): ``\(\varphi_1\)~factors over~\(\tprod\)'' should be ``\(\varphi_1\)~factors over~\(\tprod'\)''.
\item (14, 6): in the displayed equation, \(y\in F_1\) should be \(y_1\in F_1\).
\item (14, -5): \(\widetilde{E}=\bigtprod_{\alpha}E_{\alpha}\) should be \(\widetilde{E}=\bigdsum_{\alpha}E_{\alpha}\) and \(\widetilde{F}=\bigtprod_{\beta}F_{\beta}\) should be \(\widetilde{F}=\bigdsum_{\beta}F_{\beta}\).
\item (14, -1): \(\pi_{\alpha}:E\to E_{\alpha}\) should be \(\pi_{\alpha}:\widetilde{E}\to E_{\alpha}\).
\item (17, 11): in the displayed equation, it should be noted that \(x_{\alpha}\in E_{\alpha}\) and \(y_{\beta}\in F_{\beta}\).
\item (22, -9): in Corollary~III, \(L(E\tprod F;E'\tprod F)\) should be \(L(E\tprod F;E'\tprod F')\).
\item (22, -8): in Corollary~III, the claim is not shown in section~1.27, where it is also assumed that \(E'\) and~\(F'\) are finite-dimensional.
\item (25, 9): in problem~1(a), the ``only if'' part of the claim is false.
\item (30, 13), (31, 11): it is false that \(\Phi\tprod\Psi\)~is nondegenerate only if \(\Phi\) and~\(\Psi\) are nondegenerate.
\item (34, -2): it should be assumed that \(x^*\ne 0\).
\item (36, 12): \(\varphi\times\psi\) should be \((\varphi,\psi)\).
\item (36, 14): in the displayed equation, \(\dim(F;F')\) should be \(\dim L(F;F')\).
\item (37, 3): in the commutative diagram, \(\tprod\)~should be the linear map induced by~\(\tprod\).
\item (37, 16): \((\varphi\times\psi)\) should be \((\varphi,\psi)\).
\item (37, -3): \((\alpha,\beta)\) should be \(\alpha\tprod\beta\).
\item (38, 3): \(F:A\times A\to L(A;A)\) should be \(F:A\tprod A\to L(A;A)\).
\item (38, 8): \(L(A\tprod A)\) should be \(A\tprod A\).
\item (38, -5): in section~1.30, it should be assumed that \(E\ne 0\).
\item (39, 3): \(\Omega\)~should be the bilinear map induced by~\(\Omega\).
\item (40, 5): in problem~2, it should be assumed that \(E,F\ne 0\).
\item (40, -5): in problem~3, it should be assumed that \(E\)~is oriented.
\end{itemize}

\section*{Chapter~2}
\begin{itemize}
\item (42, 18): if \(A\ne 0\), it must be assumed that \(B\ne 0\) for \(\varphi\)~to be injective.
\end{itemize}

\section*{Chapter~3}
% 7 24789 8023
\begin{itemize}
\item (60, -3): it should be noted that \(v\in\bigtprod^q E\).
\item (61, 3): \(\dim E=1\) should be \(\dim E\le 1\).
\item (62, 4): in the displayed equation, it should be noted that \(u_p\in\bigtprod^p E\) and \(v_q\in\bigtprod^q E\).
\item (63, -2): in~\textbf{T}, ``unit element'' should be ``unit element~\(e\)''.
\item (64, -7): in the uniqueness theorem, proof, and remarks after, it should be noted that the homomorphisms preserve the unit.
\item (65, 1): in the proof of the uniqueness theorem, the unit element is also needed to generate~\(U\).
\item (68, 8): in problem~2, \(\lambda_1,\ldots,\lambda_p=1\) should be \(\lambda_1\cdots\lambda_p=1\).
\item (68, 9): in problem~3, it should be asked to prove that \(\tr(\varphi\psi)=\tr(\psi\varphi)\).
\item (68, -8): in the displayed equation, it should be noted that \(u^{*p}\in\bigtprod^p E^*\) and \(v_p\in\bigtprod^p E\).
\item (69, -10): the reference to~(3.8) should be deleted.
\item (72, 6): ``noncommutative'' should be ``noncommutative in general''.
\item (72, -5): \(\Phi^i_j(\omega)\)~should be~\(C^i_j(\omega)\).
\item (74, 1): in many problems in this section, the order of dual spaces should be swapped to be consistent with the rest of the chapter; for example, \(\bigtprod(E,E^*)\) should be \(\bigtprod(E^*,E)\).
\item (74, -9): in problem~7, it should be assumed that \(E\)~is finite-dimensional.
\item (74, -5): in problem~8, it should be assumed that \(E\)~is finite-dimensional.
\item (77, -3): in problem~5, the scalars \(g_{\nu\mu}=\iprod{e_{\nu}}{e_{\mu}}\) should be defined.
\item (78, -6): ``noncommutative'' should be ``noncommutative in general''.
\item (79, 6): it should be clarified how \(i_{\nu}(h)\)~acts if \(\nu>p\).
\item (80, 11): in the displayed equation, \(E^{*\nu}\)~should be~\(E^*\).
\item (82, 6): in the displayed equation, \(\langle f_1\cdot\ldots\cdot f_p\Psi\rangle\) should be \(\sprod{f_1\cdot\ldots\cdot f_p}{\Psi}\).
\item (82, -10): in the displayed equation,
\[\Psi(x_1,\ldots,x_r,x^*_1,\ldots,x^*_s)\]
should be
\[\Psi(x_{p+1},\ldots,x_{p+r},x^*_{q+1},\ldots,x^*_{q+s})\]
\item (83, -1): \(y^{*j}\)~should be~\(y^*_j\).
\end{itemize}

% References
\begin{thebibliography}{0}
\bibitem{greub1} Greub, W. \textit{Linear Algebra}, 4th~ed. Springer, 1975.
\bibitem{greub2} Greub,W. \textit{Multilinear Algebra}, 2nd~ed. Springer, 1978.
\end{thebibliography}
\end{document}
