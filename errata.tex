% Errata from Linear Algebra and Multilinear Algebra by Greub
% By John Peloquin
\documentclass[letterpaper,12pt]{article}
\usepackage{amsmath,amssymb,amsthm,enumitem,fourier,stmaryrd}

\newcommand{\Q}{\mathbb{Q}}
\renewcommand{\H}{\mathbb{H}}

\newcommand{\from}{\leftarrow}
\newcommand{\divides}{\mid}

\DeclareMathOperator{\tr}{tr}

\newcommand{\union}{\cup}
\newcommand{\after}{\circ}
\newcommand{\dsum}{\oplus}
\newcommand{\bigdsum}{\bigoplus}
\newcommand{\tprod}{\otimes}
\newcommand{\bigtprod}{\bigotimes}
\newcommand{\eprod}{\wedge}
\newcommand{\bigeprod}{\bigwedge}

\newcommand{\sprod}[2]{\langle#1,#2\rangle}
\newcommand{\iprod}[2]{(#1,#2)}
\newcommand{\oc}[1]{#1^{\perp}}
\newcommand{\proj}[1]{\overline{#1}}

% Meta
\title{Errata from \\\textit{Linear Algebra} and \textit{Multilinear Algebra}}
\author{John Peloquin}
\date{}

\begin{document}
\maketitle

% Intro
\section*{Introduction}
This document contains errata from \cite{greub1} and~\cite{greub2}. Locations within each text are indicated by coordinates~\((p,n)\), where \(p\)~is a page number and \(n\)~is a line number on page~\(p\). Positive line numbers count from the top of the page, whereas negative line numbers count from the bottom of the page. Displayed equations, diagrams, and figures are counted as single lines.

% Linear algebra
\newpage
\part*{Linear Algebra}
Errata are currently only listed for Chapters 0--V, VI (except \S~3), VII, VIII (except \S~7), XI (except \S~4--5), XII (except \S~4), and~XIII.

Minor, purely typographical inconsistencies like those between ``\(\to\)'' and ``\(\mapsto\)'', or between ``\(1\ldots n\)'' and ``\(1,\ldots,n\)'', or between ``family \(x_{\alpha}\)'' and ``family \((x_{\alpha})\)'' and ``family \(\{x_{\alpha}\}\)'', or between ``\(f\divides g\)'' and ``\(f/g\)'', are not listed below but occur frequently.

\section*{Chapter~0}
\begin{itemize}
\item (2, 8): in the definition of subgroup, ``subset~\(H\)'' should be ``nonempty subset~\(H\)''.
\item (3, -12): a homomorphism between fields must also preserve the multiplicative identity.
\item (3, -13): a subfield must also contain the multiplicative identity.
\end{itemize}

\section*{Chapter~I}
\begin{itemize}
\item (9, 16): in the definition of linear dependence, ``non-trivial linear combination of the~\(x_{\alpha}\)'' should be ``non-trivial linear combination of the~\(x_{\alpha}\) equal to zero''.
\item (12, 1): \(\lambda^i=0\) should be \(\lambda^i\ne 0\).
\item (12, 5): throughout the proof of~(ii), \(n\)~should be~\(m\).
\item (12, -5): ``The a partial order'' should be ``A partial order''.
\item (12, -4): ``chain'' should be ``nonempty chain'', and it should be noted that \(R\)~is an upper bound for the empty chain.
\item (12, -4): ``maximal element'' should be ``upper bound''.
\item (13, 8): \(x\in E\) should be \(x\in S-T\).
\item (13, 9), (13, 10): \(T\union x\) should be \(T\union\{x\}\) and \(x\union T\) should be \(\{x\}\union T\).
\item (14, 1): ``element~\(f_a\)'' should be ``elements~\(f_a\)''.
\item (14, 3), (14, 5): in the displayed equations, \(j=1\) should be \(i=1\).
\item (15, -2): in problem~10, \(\{x_{\alpha}\}_{\alpha\ne\beta}\) should be \(\{a\}\union\{x_{\alpha}\}_{\alpha\ne\beta}\).
\item (20, 16): ``\(\varphi(S)\)~is a system of generators for~\(\varphi(S)\)'' should be ``\(\varphi(S)\)~is a system of generators for~\(F\)''.
\item (22, -15), (22, -11): in part~(v) of Problem~5, the concepts of generated subspace and kernel have not yet been defined.
\item (23, 7): in the definition of subspace, ``subset of~\(E\)'' should be ``nonempty subset of~\(E\)''.
\item (27, -11): in the displayed equation, \(\lambda_i\)~should be~\(\lambda^i\) (two occurrences). Also, it should be noted that \(y_i\in E_1\) and \(z_i\in E_2\).
\item (29, -16): ``\emph{canonical projection} of~\(E\) onto~\(E_1\)'' should be ``\emph{canonical projection} of~\(E\) onto~\(E/E_1\)''.
\item (31, 14): in problem~7, ``composition'' should be ``decomposition''.
\item (31, 19): in problem~8, the period at the end of the second sentence should be a question mark.
\item (40, -8): in the second displayed equation in problem~6, \(u\in E_2'\) should be \(y\in E_2'\).
\end{itemize}

\section*{Chapter~II}
\begin{itemize}
\item (47, -6): ``assume that there'' should be ``assume that''.
\item (52, -8): \(\psi:E_1\from F\) should be \(\psi:E\from F\).
\item (52, -3): ``left inverse'' should be ``left inverse~\(\psi\)''.
\item (53, 13): ``inverse'' should be ``left inverse''.
\item (53, -1): in problem~1, the inclusion \(L(E;F)\subset C(E;F)\) is wrong.
\item (59, -5): in~(2.25), it should be noted that \(\delta_{\varrho\sigma}\)~is a Kronecker delta.\footnote{This notation is defined later in a footnote on p.~76.}
\item (62, 10): in~(2.34), \(y\)~should be~\(y_j\).
\item (63, -7): in problem~7, ``second set'' should be ``disjoint set''.
\item (67, -1): in the displayed equation, \(y^*\in F\) should be \(y^*\in F^*\).
\item (68, -7): in the displayed equation, \(\langle y^*(\varphi+\psi)x\rangle\) should be \(\sprod{y^*}{(\varphi+\psi)x}\).
\item (70, 14): ``bilinear functions'' should be ``bilinear function''.
\item (75, -2): in problem~10, the induced mapping should be
\[\proj{\varphi}^*:E^*/\oc{E_1}\from F^*/\oc{F_1}\]
\item (76, 9): ``imension'' should be ``dimension''.
\item (76, -7): in the displayed equation (and really the rest of subsection~2.31), \(\varphi^{\mu}_{\nu}\)~should be~\(\varphi^{\nu}_{\mu}\).
\item (77, -7): \(\varphi\)~should be~\(\Phi\).
\end{itemize}

\section*{Chapter~III}
\begin{itemize}
\item (83, -10): in the displayed equation, \((b^{\mu}_1\,\ldots\,b^{\mu}_n)\) should be \((\alpha^{\mu}_1\,\ldots\,\alpha^{\mu}_n)\).
\item (84, -7): ``columns of the matrix~\(\alpha^{\mu}_{\nu}\)'' should be ``columns of the matrix~\(\alpha^{*\mu}_{\nu}\)''.
\item (84, -6): \(y=y^{*\mu}\) should be \(y^*=y^{*\mu}\).
\item (85, 13): in the displayed equation, it should be noted that \(A=(\alpha^{\mu}_{\nu})\).
\item (88, 1): in the main theorem, ``system of \(n\) equations in \(m\) unknowns'' should be ``system of \(m\) equations in \(n\) unknowns''.
\item (89, 12): it should be noted that \(\dim E=n\) and \(\dim F=m\).
\item (89, -8): in the displayed equation, \(a^{\mu}_{\nu}\)~should be~\(\alpha^{\mu}_{\nu}\).
\item (91, 5): ``\(E\) automorphism of~\(E\)'' should be ``automorphism of~\(E\)''.
\item (91, 9): in the displayed equation, \(M(\varphi)^{-1}\)~should be~\(M(\varphi^{-1})\).
\item (91, -3): in problem~3, ``linear transformation'' should be ``a linear transformation''.
\item (93, -13): ``inverse of the matrix of the transformation \(x_{\nu}\to\bar{x}_{\nu}\)'' should be ``transpose of the inverse of the matrix of the transformation \(x_{\nu}\to\bar{x}_{\nu}\)''.
\item (95, 12): \(\mu=1,\ldots,n\) should be \(\mu=1,\ldots,m\).
\item (96, 1): multiplication of basis vectors by nonzero scalars should be added to the list of elementary basis transformations.
\item (96, -7): \(2\leqq\nu\leqq m\) should be \(2\leqq\nu\leqq n\).
\item (98, 3): (3.36)~should be
\[\xi^r=(\kappa^r_r)^{-1}\left(\omega^r-\sum_{\nu=r+1}^n\kappa^r_{\nu}\xi^{\nu}\right)\]
\end{itemize}

\section*{Chapter~IV}
\begin{itemize}
\item (103, 15): in~(4.6), it should be noted that \(\hat{x}_j\)~indicates that the vector~\(x_j\) is deleted.\footnote{This notation is defined later in a footnote on p.~198.}
\item (105, -8): ``Proposition~II'' should be ``Proposition~III''.
\item (106, 5): ``(4.14)'' should be ``(4.12)''.
\item (107, 13): in~(4.14), \(p=\dim E\) should be \(p=\dim E_1\).
\item (109, 1): the problem numbers on this page should be incremented by~\(1\).
\item (109, 10): in problem~6, it should be assumed that \(E\)~is real.
\item (109, -11): in problem~8, the trace of a linear transformation has not yet been defined.
\item (113, 1): ``(4.21)'' should be ``(4.22)''.
\item (113, -10): in the displayed equation, \({x^*}_i\)~should be~\(x^{*i}\).
\item (115, 1): ``(4.14)'' should be ``(4.12)''.
\item (115, 4): in the displayed equation, \(\widehat{\varphi a_j}\)~should be~\(\widehat{\varphi a_i}\).
\item (115, 9): it should be noted that \(M(\varphi)=(\alpha^{\mu}_{\nu})\).
\item (115, -10): the displayed equation should be \(\beta^i_j=\det C^j_i\).
\item (116, 11): ``(4.36)'' should be ``(4.38)''.
\item (116, -1): in the displayed matrix~\(B^j_i\), the first column should have entries \(1,\alpha^j_1,\ldots,\alpha^j_{i-1},\alpha^j_{i+1},\ldots,\alpha^j_n\).
\item (117, 3): ``(4.38)'' should be ``(4.16)'', or a reference to Problem~6.
\item (117, 7): ``(4.35) and (4.30)'' should be ``(4.38) and (4.40)''.
\item (117, -14): ``minor'' should be ``submatrix''.
\item (122, 2): on the third line of the displayed equation, \(\sigma\)~should be~\(\varrho\).
\item (130, -14): in problem~11, it should be assumed that \(E\)~is real.
\item (130, -10): in problem~12, \(\varphi:E_1\to E_1\) should be \(\varphi_1:E_1\to E_1\).
\item (132, 13): \(n\)~should be~\(E\).
\item (132, 14): ``orientated'' should be ``oriented''.
\item (135, -3): ``4.30'' should be ``4.29''.
\item (136, -6): ``4.17'' should be ``4.16'', and it should be noted that \(\xi^i_{\nu}\)~are the components of~\(x_{\nu}\) with respect to some basis.
\item (139, -13): \((a_1',b_2\ldots b_n)\) should be \((a_1',b_3\ldots b_n)\).
\item (139, -10): in the displayed equation, \(i=1\ldots n\) should be \(i=1\ldots n-1\).
\item (142, -11): in problem~5, ``isotopices'' should be ``isotopies''.
\end{itemize}

\section*{Chapter~V}
\begin{itemize}
\item (146, 1): \((\varphi,\psi)\to\psi\varphi\) should be \((\psi,\varphi)\to\psi\varphi\).
\item (148, 10): the displayed list should also include elements of the form \(asb\) for \(s\in S\) and \(a,b\in A\).
\item (148, -17): it must be assumed that \(A\ne 0\) (equivalently \(e\ne 0\)) for it to follow that \(\lambda=0\).
\item (151, 2): the extra ``can be'' should be deleted.
\item (158, 14): ``cheeked'' should be ``checked''.
\item (160, -13): ``let be'' should be ``be''.
\item (160, -8): it must be assumed that \(E\ne 0\) for \(A(E;E)^2\ne 0\).
\item (161, -1): ``non-zero, \(I\), ideal'' should be ``non-zero ideal~\(I\),''.
\item (165, 12), (166, 7): it should be clarified that \(\Gamma\) is assumed to be a field under the restrictions of the operations in \(A_{\Delta}(E;E)\).
\end{itemize}

\section*{Chapter~VI}
\begin{itemize}
\item (168, -16): it should be noted that the zero map is homogeneous of every degree (hence \emph{the} degree is not well-defined in that case).
\item (172, -4): ``product'' should be ``products''.
\item (173, -3): in problem~6, \(\deg\varphi^*=-k\).
\item (175, 4): the displayed statement should be \(xe_k\in A_{l+k}\).
\item (176, 1): \(E\)~should be~\(A\).
\item (177, 2): in the second part of problem~1, it must be assumed that \(A\ne 0\) to conclude that \(k=0\).
\item (177, 14): in problem~5, ``\(\leqq 0\) (\(\geqq 0\))'' should be ``\(\geqq 0\) (\(\leqq 0\))''.
\end{itemize}

\section*{Chapter~VII}
\begin{itemize}
\item (189, -12): in the displayed equation, \(\lambda>0\) should be \(\lambda\geqq 0\).
\item (192, 5): \(x=x_{\mu}\) should be \(y=x_{\mu}\).
\item (192, 14): ``basisvectors'' should be ``basis vectors''.
\item (193, 14): ``orthogonal bases'' should be ``orthonormal bases''.
\item (193, -8): \((\alpha^{\mu}_{\varphi})\)~should be~\((\alpha^{\mu}_{\nu})\).
\item (195, 4): in problem~3, it should be assumed that \(E\)~is finite-dimensional or else established that orthogonal projection still works as long as \(E_1\)~is finite-dimensional.
\item (204, 13): in the displayed equation, \(\sum_{\nu}g_{\nu\lambda}\xi^{\lambda}\) should be \(\sum_{\nu}g_{\nu\lambda}\xi^{\nu}\).
\item (206, 14): ``least upper bound'' should be ``least nonnegative upper bound'' to account for the case \(E=0\) where there are no unit vectors.
\item (207, 7): ``naturaltopology'' should be ``natural topology''.
\item (210, 15): throughout subsection~7.24, it should be assumed that \(A\ne 0\).
\end{itemize}

\section*{Chapter~VIII}
\begin{itemize}
\item (217, 13): the concept of a metric tensor has not yet been defined.
\item (217, 14): in the displayed equation, \(\displaystyle\sum_{\nu}\)~should be~\(\displaystyle\sum_{\lambda}\).
\item (220, -1): in problem~1, equality holds only if \(\psi=\lambda\varphi\) or \(\varphi=\lambda\psi\).
\item (223, -2): in~(8.22), \(\lambda-\lambda_i\) should be \(\lambda_i-\lambda\).
\item (226, 17): in problem~11, the concept of a rotation has not yet been defined.
\item (227, 14): \(E_1\in\oc{E_2}\) should be \(E_1\subset\oc{E_2}\).
\item (231, 7): the proof of the normal form~(8.35) is incorrect because it is not true in general that the~\(a_n\) defined form an orthonormal basis of the space.
\item (231, -5): ``Proposition~II'' should be ``Proposition~III''.
\item (234, 4): in the displayed equation, \(a^{\kappa}_{\mu}\)~should be~\(\alpha^{\kappa}_{\mu}\).
\item (234, -16): ``\emph{rotation}'' should be ``\emph{rotation}~\(\varphi\)''.
\item (234, -10): it must be assumed that \(e\ne 0\).
\item (234, -9): ``sec.~4, 17'' should be ``sec.~4.17''.
\item (238, -7): in~(8.40), the second equation should be \(\sin\theta=-\tfrac{1}{2}\tr(j\after\varphi)\).
\item (239, 17): in the displayed equation, ``\(\equiv\)''~should be~``\(=\)''.
\item (240, -13): ``see~8.21'' should be ``sec.~8.21''.
\item (241, 6): in~(8.49), \(x\)~should be~\(u\).
\item (242, -13): in the displayed equation, ``\(\equiv\)''~should be~``\(=\)''.
\item (243, 7): \(E\)~should be~\(e\).
\item (243, -5): ``see~8.21'' should be ``sec.~8.23''.
\item (243, -4): in the displayed equation, \((z,\tau,z)\) should be~\(\iprod{z}{\tau z}\).
\item (243, -2): it should be noted that \(\Delta\)~is the positive normed determinant function in~\(E_1\).
\item (244, -11): \(F_1\)~should be~\(F\).
\item (245, 8): in the displayed equation, \(b\)~should be~\(b_1\).
\item (245, -10), (245, -5): ``(8.53)'' should be ``(8.54)''.
\item (246, 10), (246, -10): in problems 3 and~5, ``plane'' should be ``plane~\(E\)''.
\item (247, -1): \(f\)~should be defined so that \(f(0)=0\).
\item (248, -8): in problem~16, \(E\)~should be~\(\H\).
\end{itemize}

\section*{Chapter~XI}
\begin{itemize}
\item (328, 14), (328, -7): the proof of the equality condition for the triangle inequality does not properly account for the case \(x=0\); in particular, ``\(y=\lambda x\)'' should be ``\(y=\lambda x\) or \(x=\lambda y\)''.
\item (332, 11): in problem~3, the displayed equation should be
\[\iprod{z_1}{z_2}=\bigl(\iprod{x_1}{x_2}+\iprod{y_1}{y_2}\bigr)+i\bigl(\iprod{x_2}{y_1}-\iprod{x_1}{y_2}\bigr)\]
and it should be noted that \(z_k=x_k+iy_k\).
\item (334, -7): ``automorphism'' should be ``involution''.
\item (335, -12): in the displayed equation, \(\Phi\)~should be~\(\varphi\).
\item (335, -5): the concept of a conjugate determinant function is not defined.
\item (335, -3): in the displayed equation, \(\bar{x}_p\)~should be~\(\bar{x}_n\).
\item (336, -8): in the displayed equation, \(\displaystyle\sum_{\mu,\nu}\)~should be~\(\displaystyle\sum_{\mu,\lambda}\).
\item (337, -16): ``normal mapping'' should be ``normal mappings''.
\item (337, -10): in the displayed equation, \(\lambda\)~should be~\(\lambda_1\).
\item (339, 8): it must be assumed that \(e\ne 0\).
\item (339, 14): in problem~1, ``bilinear'' should be ``binary''.
\end{itemize}

\section*{Chapter~XII}
\begin{itemize}
\item (352, -11): the degree(s) of the zero polynomial should be clarified.
\item (353, -8): \(A\in\Gamma[t]\) should be \(A=\Gamma[t]\).
\item (356, 3): in problem~1(d), it is false that \(f_1(g)=f_2(g)\) implies \(f_1=f_2\).
\item (356, 9): in problem~2, it is false that if \(\int_1\) and~\(\int_2\) are two linear mappings on~\(\Gamma[t]\) with \(d\after\int_1=\iota=d\after\int_2\) then there is a fixed \(\alpha\in\Gamma\) with \((\int_1-\int_2)f=\alpha\) for all \(f\in\Gamma[t]\).
\item (358, 6): ``second polynomial'' should be ``second monic polynomial''.
\item (358, 10): in subsection~12.6, greatest common divisor and relative primality should be defined for arbitrary nonzero polynomials, and the rest of the chapter updated to reflect this.
\item (358, -13): irreducible polynomials should be required to be nonconstant, and the rest of the chapter updated to reflect this.
\item (358, -12): ``ar'' should be ``or''.
\item (358, -4): in subsection~12.7, least common multiple should be defined for arbitrary nonzero polynomials and the rest of the chapter updated to reflect this.
\item (360, -5): \(g\leqq 0\) should be \(0\leqq g\).
\item (361, 3): it should be clarified that \(k_i\ge 1\).
\item (363, 8): in the proof of Proposition~IV, it must be assumed (without loss of generality) that \(f\) and~\(g\) are monic.
\item (363, -4): ``common factor'' should be ``nonconstant common factor''.
\item (364, 10): in the displayed equation, \(n\)~should be~\(r\).
\item (366, -3): ``commutative'' should be deleted.
\item (367, 14), (367, 16): in Examples I and~II, it must be assumed that \(A\ne 0\).
\item (367, -13): in the displayed equation, \(\Phi(f)\) and~\(\pi(f)\) should be swapped.
\item (367, -3): ``Proposition~V'' should be ``Proposition~I''.
\item (369, 8), (369, 13): the~Q's should all be~\(\Q\)'s.
\end{itemize}

\section*{Chapter~XIII}
\begin{itemize}
\item (387, 5): ``Proposition~II'' should be ``Proposition~I''.
\item (387, 7): the displayed equation should be
\[K(f)=K(1)=0\]
\item (387, 15): ``non-zero'' should be deleted.
\item (388, -11): \(\varphi_0\)~should be~\(\varphi\).
\item (389, 4): in problem~2, ``Given'' should be ``Give''.
\item (389, -15): in problem~6, ``minimal polynomial'' should be ``minimum polynomial''.
\item (389, -11): in problem~6, in the displayed equation, it should be clarified that the~\(\lambda_{\nu}\) are not all zero.
\item (390, 8): in problem~8, in the displayed equation, \(\alpha_v\)~should be~\(\alpha_{\nu}\).
\item (390, -13): in problem~13, \(L(E,E)\)~should be~\(L(E;E)\).
\item (393, -7): ``vektor'' should be ``vector''.
\item (395, 14): in problem~1, ``stable subspace'' should be ``non-zero stable subspace''.
\item (396, -17): in problem~7, ``unique'' should be deleted.
\item (397, -1): ``\(f\) irreducible'' should be ``\(f\) monic irreducible''.
\item (398, 4): \(a\in K(f)\) should be \(a\in K(h)\).
\item (398, 7): ``Thus \(K(g)=K_{\mu}\). Now it follows from (13.28) that \(a\in K_{\mu}\).'' should be ``Thus \(g=\mu\). Now it follows that \(h\in I_{\mu}\).''
\item (398, 10): it should be clarified that this is the prime decomposition of~\(\mu\) (so the \(f_i\) are distinct monic and \(k_i\ge 1\)).
\item (400, -3): ``Proposition~III'' should be ``Proposition~II''.
\item (402, 12): it should be clarified that \(m=\deg\mu\). Also, for this reason, \(\mu\)~should not be used as the index variable!
\item (402, 13): \(1\dots m-1\) should be \(0\dots m-1\).
\item (402, -13): \(\pi^*\)~should be~\(\pi\).
\item (403, 11): in the displayed equation, \(F_i\)~should be~\(F_j\).
\item (403, 16): in the displayed equation, \(F_r\)~should be~\(F_j\), and it should be clarified that \(\dim F_j>0\).
\item (403, -15): \(j=1\) should be \(r=1\).
\item (403, -12): ``\(\varphi\)~is cyclic'' should be ``\(E\)~is cyclic''.
\item (403, -8): in part~(i) of Theorem~I, it should be clarified that this is the prime decomposition of~\(\mu\) (so \(f\)~is monic and \(k\ge 1\)).
\item (404, 10): \(\mu\divides f^{k_i}\) should be \(\mu\divides f^{k_1}\).
\item (404, 13): ``13.10'' should be ``13.11''.
\item (404, -1): ``\(f\)~is an irreducible'' should be ``\(f\)~is a monic irreducible''.
\item (406, 3): ``irreducible'' should be deleted.
\item (406, 5): ``13.12'' should be ``13.13''.
\item (407, 5): in the matrix, there should be a~\(0\) in the upper right corner.
\item (407, -16): ``13.12'' should be ``13.13''.
\item (407, -11): it should be clarified that this is the prime decomposition of~\(\mu\) (so \(f\)~is monic and \(k\ge 1\)).
\item (407, -10): ``irreducible subspaces'' should be ``non-zero irreducible subspaces''.
\item (407, -8): ``13.12'' should be ``13.14''.
\item (410, -10): \(\psi\)~should be~\(\psi_i^j\).
\item (412, -10): it should be clarified that this is the prime decomposition of~\(f\) (so the~\(f_i\) are distinct monic and \(m_i\ge 1\)).
\item (413, 9): in problem~1, ``\(\varphi\in A(E;E)\)'' should be moved from~(i) to~(ii).
\item (414, 3): in problem~6(a), ``\(\varphi_j\)~is cyclic'' should be ``\(F_j\)~is cyclic''.
\item (414, 18): in problem~8(a), \(K(f^{k-1})\)~should be~\(K(f^k)\).
\item (416, -9): it should be clarified that \(j_1\ne j_2\).
\item (417, 5): the first displayed equation should be
\[f_1^{l_2}(\varphi)f_1^{l_1-l_2}(\varphi)y_1=0\]
\item (418, 5): \(\tau\)~need not be a surjection.
\item (419, 5): \(f\divides\mu\) should be \(\mu\divides f\).
\item (420, 13): \(F_0\)~should be~\(F\).
\item (422, 5): ``subalgebra of (cf. example~I, see~5.2)'' should be ``subalgebra of \(A(E;E)\) (cf. example~I, sec.~5.2)''.
\item (422, -2): \(\varphi\)~should be~\(\varphi_i\).
\item (423, 5): \(E_j\)~should be~\(F_j\).
\item (425, -6): ``13.14'' should be ``13.15''.
\item (427, 15): ``12.17'' should be ``12.16''.
\item (427, 16): ``decompositions'' should be ``decomposition''.
\item (427, -7): ``12.14'' should be ``12.13''.
\item (428, 2): it should be clarified that this is the prime decomposition of~\(\mu\) (so the~\(f_i\) are monic).
\item (428, -14): the displayed equation should be
\[E=F_1\dsum\cdots\dsum F_s\]
\item (429, 3): \(g\)~should be~\(q\).
\item (429, 5): ``over~\(\Delta[t]\)'' should be ``in~\(\Delta[t]\)''.
\item (429, -9), (429, -8): in Theorem~I, it should be clarified that the prime decomposition of the minimum polynomial of~\(\varphi\) is \(f_1^{k_1}\cdots f_r^{k_r}\).
\item (429, -3): \(\psi_N=\psi_N\) should be \(\psi_N=\varphi_N\).
\item (430, 12): ``see.~13.15'' should be ``sec.~13.14''.
\item (431, 4): ``polynomial'' should be ``polynomials''.
\item (432, 3): \(\varphi_i:E\to E\) should be \(\varphi_i:E_i\to E_i\).
\item (432, 5): \(E\)~should be~\(E_i\).
\item (432, 7): in the displayed equation, \(E\)~should be~\(E_i\) (twice).
\item (432, -1): it should be clarified that this is the prime decomposition of~\(\mu_1\) (so the~\(f_i\) are monic).
\item (437, -11), (439, 8): in Theorems I and~II, the definition of \emph{homothetic} must be loosened slightly to allow for possibly negative scalars.
\end{itemize}

% Multilinear algebra
\newpage
\part*{Multilinear Algebra}
Errata are currently only listed for Chapters 1, 2 (except \S~4,8--17), 3--4, and 5 (except \S~15--28). Minor purely typographical inconsistencies are not listed.

\section*{Chapter~1}
\begin{itemize}
\item (4, 11): in problem~3, it should be assumed that \(E\ne 0\).
\item (4, -7): in problem~5(b), the claim is false.
\item (7, -5): in the displayed equation, \(\lambda^{\nu}_{\alpha}\in\Gamma\) should be \(\lambda^{\alpha}_{\nu}\in\Gamma\).
\item (11, -10): ``satisfis'' should be ``satisfies''.
\item (11, -3): The function name~\(f\) should not be reused.
\item (12, 9): in problem~1,
\[(\xi^1,\ldots,\xi^n)\times(\eta^1,\ldots,\eta^m)\]
should be
\[\bigl(\,(\xi^1,\ldots,\xi^n),(\eta^1,\ldots,\eta^m)\,\bigr)\]
\item (12, -6): in problem~6(a), the tensor product of linear maps has not yet been defined.
\item (13, 15): ``\(\varphi_1\)~factors over~\(\tprod\)'' should be ``\(\varphi_1\)~factors over~\(\tprod'\)''.
\item (14, 6): in the displayed equation, \(y\in F_1\) should be \(y_1\in F_1\).
\item (14, -5): \(\widetilde{E}=\bigtprod_{\alpha}E_{\alpha}\) should be \(\widetilde{E}=\bigdsum_{\alpha}E_{\alpha}\) and \(\widetilde{F}=\bigtprod_{\beta}F_{\beta}\) should be \(\widetilde{F}=\bigdsum_{\beta}F_{\beta}\).
\item (14, -1): \(\pi_{\alpha}:E\to E_{\alpha}\) should be \(\pi_{\alpha}:\widetilde{E}\to E_{\alpha}\).
\item (17, 11): in the displayed equation, it should be noted that \(x_{\alpha}\in E_{\alpha}\) and \(y_{\beta}\in F_{\beta}\).
\item (22, -9): in Corollary~III, \(L(E\tprod F;E'\tprod F)\) should be \(L(E\tprod F;E'\tprod F')\).
\item (22, -8): in Corollary~III, the claim is not shown in subsection~1.27, where it is also assumed that \(E'\) and~\(F'\) are finite-dimensional.
\item (24, 5): in the displayed equation, \(\widetilde{\psi}\tprod\psi\) should be \(\widetilde{\psi}\after\psi\).
\item (25, 9): in problem~1(a), the ``only if'' part of the claim is false.
\item (30, 13), (31, 11): it is false that \(\Phi\tprod\Psi\)~is nondegenerate only if \(\Phi\) and~\(\Psi\) are nondegenerate.
\item (34, -2): it should be assumed that \(x^*\ne 0\).
\item (36, 12): \(\varphi\times\psi\) should be \((\varphi,\psi)\).
\item (36, 14): in the displayed equation, \(\dim(F;F')\) should be \(\dim L(F;F')\).
\item (37, 3): in the commutative diagram, \(\tprod\)~should be the linear map induced by~\(\tprod\).
\item (37, 16): \((\varphi\times\psi)\) should be \((\varphi,\psi)\).
\item (37, -3): \((\alpha,\beta)\) should be \(\alpha\tprod\beta\).
\item (38, 3): \(F:A\times A\to L(A;A)\) should be \(F:A\tprod A\to L(A;A)\).
\item (38, 8): \(L(A\tprod A)\) should be \(A\tprod A\).
\item (38, -5): in subsection~1.30, it should be assumed that \(E\ne 0\).
\item (39, 3): \(\Omega\)~should be the bilinear map induced by~\(\Omega\).
\item (40, 5): in problem~2, it should be assumed that \(E,F\ne 0\).
\item (40, -5): in problem~3, it should be assumed that \(E\)~is oriented.
\end{itemize}

\section*{Chapter~2}
\begin{itemize}
\item (42, 18): if \(A\ne 0\), it must be assumed that \(B\ne 0\) for \(\varphi\)~to be injective.
\end{itemize}

\section*{Chapter~3}
\begin{itemize}
\item (60, -3): it should be noted that \(v\in\bigtprod^q E\).
\item (61, 3): \(\dim E=1\) should be \(\dim E\le 1\).
\item (62, 4): in the displayed equation, it should be noted that \(u_p\in\bigtprod^p E\) and \(v_q\in\bigtprod^q E\).
\item (63, -2): in~\textbf{T}, ``unit element'' should be ``unit element~\(e\)''.
\item (64, -7): in the uniqueness theorem, proof, and remarks after, it should be noted that the homomorphisms preserve the unit.
\item (65, 1): in the proof of the uniqueness theorem, the unit element is also needed to generate~\(U\).
\item (68, 8): in problem~2, \(\lambda_1,\ldots,\lambda_p=1\) should be \(\lambda_1\cdots\lambda_p=1\).
\item (68, 9): in problem~3, it should be asked to prove that \(\tr(\varphi\psi)=\tr(\psi\varphi)\).
\item (68, -8): in the displayed equation, it should be noted that \(u^{*p}\in\bigtprod^p E^*\) and \(v_p\in\bigtprod^p E\).
\item (69, -10): the reference to~(3.8) should be deleted.
\item (72, 6): ``noncommutative'' should be ``noncommutative in general''.
\item (72, -5): \(\Phi^i_j(\omega)\)~should be~\(C^i_j(\omega)\).
\item (74, 1): in many problems in this subsection, the order of dual spaces should be swapped to be consistent with the rest of the chapter; for example, \(\bigtprod(E,E^*)\) should be \(\bigtprod(E^*,E)\).
\item (74, -9): in problem~7, it should be assumed that \(E\)~is finite-dimensional.
\item (74, -5): in problem~8, it should be assumed that \(E\)~is finite-dimensional.
\item (77, -3): in problem~5, the scalars \(g_{\nu\mu}=\iprod{e_{\nu}}{e_{\mu}}\) should be defined.
\item (78, -6): ``noncommutative'' should be ``noncommutative in general''.
\item (79, 6): it should be clarified how \(i_{\nu}(h)\)~acts if \(\nu>p\).
\item (79, -8), (79, -6): the definitions of the operators \(i_A(h)\) and~\(i_S(h)\) given in this subsection are inappropriate for use with the algebras of skew-symmetric and symmetric multilinear functions, respectively. See below.
\item (80, 11): in the displayed equation, \(E^{*\nu}\)~should be~\(E^*\).
\item (82, 6): in the displayed equation, \(\langle f_1\cdot\ldots\cdot f_p\Psi\rangle\) should be \(\sprod{f_1\cdot\ldots\cdot f_p}{\Psi}\).
\item (82, -10): in the displayed equation,
\[\Psi(x_1,\ldots,x_r,x^*_1,\ldots,x^*_s)\]
should be
\[\Psi(x_{p+1},\ldots,x_{p+r},x^*_{q+1},\ldots,x^*_{q+s})\]
\item (83, -1): \(y^{*j}\)~should be~\(y^*_j\).
\end{itemize}

\section*{Chapter~4}
\begin{itemize}
\item (85, 1): it should be assumed that \(i<j\).
\item (87, 12): in the displayed equation, \(u=x^{*1}\tprod\cdots\tprod x^{*p}\) should be \(u^*=x^{*1}\tprod\cdots\tprod x^{*p}\).
\item (90, 6): ``restriction of~\(\pi_X\)'' should be ``restriction of~\(\pi_A\)''.
\item (91, 2): in the displayed equation, \(X(E)\)~should be~\(X^p(E)\).
\item (93, 11): in~(4.31), \(\pi_r v\)~should be~\(\pi_S v\).
\item (95, -5): in~(4.36), \(\bigtprod E^*\)~should be~\(\bigtprod^p E^*\) and \(\bigtprod E\)~should be~\(\bigtprod^p E\).
\end{itemize}

\section*{Chapter~5}
\begin{itemize}
\item (97, 15): it should be assumed that \(i<j\).
\item (99, -15): in problem~2(a), in the displayed equation, \(x_p\)~should be~\(x_n\).
\item (99, -10): in problem~2(b), in the definition of~\(\varphi\), \((-1)^{n-1}\) should be \((-1)^{n-i}\).
\item (100, -5): ``\(p\)-linear mappings'' should be ``skew-symmetric \(p\)-linear mappings''.
\item (101, -2): ``\(p\)th exterior algebra'' should be ``\(p\)th exterior power''.
\item (105, -13): it should be assumed that \((\epsilon x)^2=0\) for all \(x\in E\).
\item (106, 4): the codomain of~\(g\) should be~\(\bigwedge E^*\).
\item (108, 3): in problem~3, the matrix of a 2-vector has not been defined.
\item (108, 4): in problem~4, the rightmost matrix in the Lagrange identity should be
\[
\begin{pmatrix}
\eta^1_{\nu_1}&\cdots&\eta^1_{\nu_p}\\
\vdots&&\vdots\\
\eta^p_{\nu_1}&\cdots&\eta^p_{\nu_p}
\end{pmatrix}
\]
Also, in the hint, the order of the two factors in the scalar product should be swapped, to be consistent with the rest of the chapter.
\item (108, -2): \(\varphi_{\eprod},\bigeprod E\to\bigeprod F\) should be \(\varphi_{\eprod}:\bigeprod E\to\bigeprod F\).
\item (115, 5): it should be clarified that \(\omega_E=\omega\) and \(\Omega_E=\Omega\).
\item (115, -4): in problem~1(b), in the displayed equation, \(\pi_F\)~should be~\(\pi_E\).
\item (116, -8): in problem~8, the transformations \((\psi-\lambda\iota)_{\eprod}\) and \(\psi_i\) should all be restricted to~\(\bigeprod^n E\). Also, \(\psi_0=(-1)_n\iota\) should be \(\psi_0=(-1)^n\iota\).
\item (117, 8): it must be assumed that the dual map exists.
\item (118, 13): \(\Omega(h)\)~should be~\(\Omega_h\).
\item (119, 8): in the proof of Proposition~5.14.2, it should be assumed that \(v\in\bigeprod^p E\).
\item (119, -4): in problem~3, it should be assumed that \(e_{\nu},e^{*\nu}\) is a pair of dual bases of \(E,E^*\).
\item (120, 7): in problem~6, in the displayed equation, the order of the two factors in the scalar product should be swapped to be consistent with the rest of the chapter.
\item (141, -8): in Proposition~5.30.1(B), \(A(E)\)~has not been defined.
\item (141, -1): in the proof of Proposition~5.30.1(B), the first and second equalities in the first displayed equation should be swapped.
\item (142, -1): in the third equality of the displayed equation, \(\theta^T(\varphi)\cdot\Psi\) should be \(\theta^T(\varphi)\Phi\cdot\Psi\).
\item (143, 1): in subsection~5.32, the substitution operator~\(i_A(h)\) should be redefined on~\(T^p(E)\) as
\[i_A(h)=\frac{1}{p}\sum_{\nu=1}^p(-1)^{\nu-1}i_{\nu}(h)\]
The missing factor of~\(1/p\) introduces errors throughout the subsection.
\item (143, -3): in the displayed equation, the tensor product symbols should be function product symbols.
\item (145, 4): in the displayed equation, \(i(h)\)~should be~\(i_A(h)\).
\item (145, -11): in the heading of subsection~5.33, \(T^{\bullet}(E)\)~should be~\(A^{\bullet}(E)\).
\item (146, -16): ``mappings'' should be ``functions''.
\item (146, -8): in the displayed equation, \(A^p(E)\)~should be~\(T^p(E)\) and \(A_p(E)\)~should be~\(T_p(E)\).
\item (147, 9): the proof should be
\begin{align*}
\sprod{\Phi}{x_1\eprod\cdots\eprod x_p}_A&=\frac{1}{p!}\sprod{\Phi}{x_1\eprod\cdots\eprod x_p}\\
	&=\sprod{\Phi}{A(x_1\cdot\ldots\cdot x_p)}\\
	&=\sprod{A\Phi}{x_1\cdot\ldots\cdot x_p}\\
	&=\sprod{\Phi}{x_1\cdot\ldots\cdot x_p}\\
	&=\Phi(x_1,\ldots,x_p)
\end{align*}
\item (147, -1): in the displayed equation, \(A(x_1\tprod\cdots\tprod x_p)\) should be \(A(x_1\cdot\ldots\cdot x_p)\) and \(f_1(x_{\sigma(p)})\)~should be~\(f_1(x_{\sigma(1)})\).
\end{itemize}

% References
\begin{thebibliography}{0}
\bibitem{greub1} Greub, W. \textit{Linear Algebra}, 4th~ed. Springer, 1975.
\bibitem{greub2} Greub,W. \textit{Multilinear Algebra}, 2nd~ed. Springer, 1978.
\end{thebibliography}
\end{document}
